\documentclass{ib002}
\usepackage[utf8]{inputenc}
\usepackage{scrextend}
\usepackage[T1]{fontenc}
\usepackage{lmodern}
\usepackage{paralist}

%\assignment{}
%\points{20 }
\duedate{7.\,5.\,2019}
%\name{}
%\uco{}

\setlength{\parindent}{0cm}
\setlength{\parskip}{3mm plus2pt minus2pt}

\begin{document}

\newcommand\N{\mathbb{N}}
\newcommand\Z{\mathbb{Z}}
\newcommand\R{\mathbb{R}}
\newcommand\bigO[1]{\mathcal{O}(#1)}
\newcounter{TaskCounter}

%%%%%%%%%%%%%%%%%%%%%%%%%%%%%%%%%%%%%%%%%%%%%%%%%%%%%%%%%%%%%%%%%%%%%%%%%%%%%%%
\stepcounter{TaskCounter}
\exercise{\arabic{TaskCounter}: 15 bodů; příklad \stepcounter{TaskCounter}\arabic{TaskCounter}: 20 bodů}
\textbf{[Příklad 1]}
Zapište algoritmus, který je parciálně korektní vzhledem k~libovolným vstupním
a~výstupním podmínkám.

%%%%%%%%%%%%%%%%%%%%%%%%%%%%%%%%%%%%%%%%%%%%%%%%%%%%%%%%%%%%%%%%%%%%%%%%%%%%%%%
\vspace{9cm}
\textbf{[Příklad 2]}
Do tabulky doplňte symboly P a N podle toho, zda platí (P) nebo neplatí (N), že
$3n^2 + 5n \in \mathit{\Omega}(f)$, resp. $3n^2 + 5n \in \bigO{f}$ a $3n^2 + 5n
\in \mathit{\Theta}(f)$, pro funkci $f$ zadanou v~prvním sloupci tabulky.

\bigskip
\begin{center}
\begin{tabular}{|c|c|c|c|}
	\hline  & & & \\[1.5ex]
	$f$ & $3n^2 + 5n \in \mathit{\Omega}(f)$
	  & $3n^2 + 5n \in \bigO{f}$
	  & $3n^2 + 5n \in \mathit{\Theta}(f)$ \\[1.5ex] 
	\hline \hline  & & & \\[1.5ex]
	$n$ & \ &\ & \\ [1.5ex] \hline & & & \\[1.5ex]
	$n^2$ & \ &\ & \\  [1.5ex] \hline & & & \\[1.5ex]
	$n^2\log n$ & \ &\ & \\[1.5ex] \hline & & & \\[1.5ex]
	$n^{3}$ & \ &\ & \\ [1.5ex] \hline & & & \\[1.5ex]
	$3^n$ & \ &\ & \\ [1.5ex] \hline %& & & \\[1.5ex]
\end{tabular}
\end{center}

%%%%%%%%%%%%%%%%%%%%%%%%%%%%%%%%%%%%%%%%%%%%%%%%%%%%%%%%%%%%%%%%%%%%%%%%%%%%%%%
\newpage
\stepcounter{TaskCounter}
\exercise{\arabic{TaskCounter}, 25 bodů}
Mějme pole $A[1 \dots n]$ seřazených přirozených čísel, které bylo cyklicky
posunuto o~$k$ pozic doprava. Například pole $[35, 42, 5, 15, 27, 29]$ je pole,
které bylo cyklicky posunuto o~$k = 2$ pozic, zatímco $[27, 29, 35, 42, 5, 15]$
bylo posunuto o~$k = 4$ pozic. V libovolném poli dokážeme najít maximální prvek
v~čase $\bigO{n}$. Pokud známe $k$, najdeme maximální prvek z~pole $A$ v~čase
$\bigO{1}$.
\begin{compactenum}
	\item Navrhněte algoritmus, který najde maximum v~čase $\bigO{\log(n)}$,
		pokud neznáme $k$.
	\item Navrhněte algoritmus, který vyhledá zadanou hodnotu v~poli v~čase
		$\bigO{\log(n)}$ (pokud neznáme $k$).
\end{compactenum}

%%%%%%%%%%%%%%%%%%%%%%%%%%%%%%%%%%%%%%%%%%%%%%%%%%%%%%%%%%%%%%%%%%%%%%%%%%%%%%%
\newpage
\stepcounter{TaskCounter}
\exercise{\arabic{TaskCounter}, 25 bodů}
Dokažte, že graf obsahující kružnici liché délky nemůže být bipartitní.

%%%%%%%%%%%%%%%%%%%%%%%%%%%%%%%%%%%%%%%%%%%%%%%%%%%%%%%%%%%%%%%%%%%%%%%%%%%%%%%
%\newpage
%\stepcounter{TaskCounter}
%\exercise{\arabic{TaskCounter}, 25 bodů}

%Je operace \texttt{Delete} v~binárním vyhledávácím stromu komutativní – smazání uzlu
%$x$ a pak $y$ dává stejný výsledek jako smazání uzlu $y$ a pak $x$? Pokud ano,
%dokažte to, jinak ukažte protipříklad.

\end{document}
