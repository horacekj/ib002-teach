\documentclass[12pt,a4paper,landscape]{article}
\usepackage[czech]{babel}
\usepackage[utf8]{inputenc}
\usepackage[T1]{fontenc}
\usepackage{multicol}
\usepackage{enumitem}
\usepackage{lmodern}
\usepackage{graphicx}
\textwidth 26.6cm \textheight 19cm
\topmargin -2.5cm
\oddsidemargin -1cm

\begin{document}
\pagestyle{empty}

\setlength{\parindent}{0pt}
\setlength{\columnsep}{20pt}
\setlist[enumerate]{leftmargin=*}

\Large

\begin{multicols}{2}
\begin{enumerate}
\item Iterativní algoritmus
\item Rekurzivní algoritmus
\item Rozděl a panuj
\item Složitost rekurzivního algoritmu
\item Korektnost rekurzivního algoritmu
\item Strom rekurzivních volání
\item Master theorem
\item Substituční metoda
\item Metoda stromu
\end{enumerate}
\end{multicols}

\rule{\linewidth}{1pt}

\begin{multicols}{2}
\begin{enumerate}[label=\Alph*]

\item zobrazuje jednotlivá rekurzivní volání.

\item je přístup dělení problému na menší podproblémy.
Sjednocením částečných řešení vyřešíme původní problém. Takové
algoritmy se často volají rekurzivně.
\item je metoda odvozování složitosti rekurzivních algoritmů založená na
jednoduchém dosazení do vzorce..
\item nejčastěji dokazujeme pomocí matematické indukce vedené vzhledem k počtu
zanoření rekurze.
\item nejsnadněji zapíšeme pomocí rekurentní rovnice, která vyjadřuje složitost
výpočtu na vstupu velikosti $n$ pomocí složitosti výpočtů na vstupech menší
velikosti.
\item je takový, který spočívá v~opakování určité své části (bloku).
\item Zkonstruujeme strom rekurzivních volání, jehož vrcholy ohodnotíme složitostí
výpočtu na příslušné úrovni rekurze (tj. bez rekurzivně volaných podčástí).
Výsledná složitost je součtem ohodnocení všech vrcholů stromu.
\item Uhodneme řešení a dokážeme správnost matematickou indukcí.
\item opakuje kód prostřednictvím volání sebe sama (obvykle na podproblémech menší
velikosti). Každý rekurzivní algoritmus lze převést do iterativní podoby.
\end{enumerate}

\end{multicols}

\end{document}
