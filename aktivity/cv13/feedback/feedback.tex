\documentclass[12pt,a5paper]{article}
\usepackage[czech]{babel}
\usepackage[utf8]{inputenc}
\usepackage[T1]{fontenc}
\usepackage{geometry}
\usepackage{lmodern}
\usepackage{multicol}
\usepackage{listings}
\textwidth 128mm \textheight 210mm
\topmargin -3cm
\oddsidemargin -1.5cm
\begin{document}
\pagestyle{empty}

\section*{Zpětná vazba (na cvičení)}
Tady a teď máte jedinečnou příležitost změnit předmět IB111 a~udělat jej pro
budoucí generace lepší. Využijte této příležitosti a~zamyslete se nad tím, co
se vám v~předmětu líbilo, nelíbilo, co zachovat a~co změnit, a~napište nám to.
Více návodných otázek najdete na zadní straně, můžete se jimi inspirovat.

\newpage

\section*{Zpětná vazba (na cvičení)}
Tady a teď máte jedinečnou příležitost změnit předmět IB111 a~udělat jej pro
budoucí generace lepší. Využijte této příležitosti a~zamyslete se nad tím, co
se vám v~předmětu líbilo, nelíbilo, co zachovat a~co změnit, a~napište nám to.
Více návodných otázek najdete na zadní straně, můžete se jimi inspirovat.

\newpage

\section*{Otázky}

Následující otázky můžete chápat jako inspiraci pro psaní zpětné vazby na
druhou stranu. Rozhodně není třeba je jich striktně držet.

\begin{enumerate}
	\item Jak by vypadalo tvé ideální cvičení (např. míra opakování tématu,
		  programování, demonstrativních ukázek cvičícím na projektoru, ...)?
	\item Které cvičení se ti líbilo nejvíc a~které nejméně (a~proč)?
	\item Co si myslíš o~párovém programování? Jaké to pro tebe mělo
		  výhody/nevýhody oproti samostatnému programování? Byl bys radši,
		  kdyby se na cvičeních programovalo více samostatně?
	\item Měl jsi vždy všechny potřebné informace, komunikoval s~tebou cvičící?
	\item Vyhovoval ti web cvičení jako platforma pro předávání informací?
		  Jak poctivě jsi plnil přípravy na hodinu?
	\item Domácí úlohy. Styl? Náročnost? Pracnost?
	\item Vnitra. Styl? Náročnost? Pracnost?
	\item Která část předmětu ti přišla nejtěžší (nebo třeba nejhůře
	      vysvětlená, příliš málo úloh ve sbírce, ...)?
	\item Co je ta nejlepší věc, kterou sis ze cvičení odnesl?
\end{enumerate}

\newpage

\section*{Otázky}

Následující otázky můžete chápat jako inspiraci pro psaní zpětné vazby na
druhou stranu. Rozhodně není třeba je jich striktně držet.

\begin{enumerate}
	\item Jak by vypadalo tvé ideální cvičení (např. míra opakování tématu,
		  programování, demonstrativních ukázek cvičícím na projektoru, ...)?
	\item Které cvičení se ti líbilo nejvíc a~které nejméně (a~proč)?
	\item Co si myslíš o~párovém programování? Jaké to pro tebe mělo
		  výhody/nevýhody oproti samostatnému programování? Byl bys radši,
		  kdyby se na cvičeních programovalo více samostatně?
	\item Měl jsi vždy všechny potřebné informace, komunikoval s~tebou cvičící?
	\item Vyhovoval ti web cvičení jako platforma pro předávání informací?
		  Jak poctivě jsi plnil přípravy na hodinu?
	\item Domácí úlohy. Styl? Náročnost? Pracnost?
	\item Vnitra. Styl? Náročnost? Pracnost?
	\item Která část předmětu ti přišla nejtěžší (nebo třeba nejhůře
	      vysvětlená, příliš málo úloh ve sbírce, ...)?
	\item Co je ta nejlepší věc, kterou sis ze cvičení odnesl?
\end{enumerate}

\end{document}
