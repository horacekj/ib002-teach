\documentclass[12pt,a5paper]{article}
\usepackage[czech]{babel}
\usepackage[utf8]{inputenc}
\usepackage[T1]{fontenc}
\usepackage{geometry}
\usepackage{lmodern}
\usepackage{multicol}
\usepackage{listings}
\textwidth 128mm \textheight 210mm
\topmargin -3cm
\oddsidemargin -1.5cm
\begin{document}
\pagestyle{empty}

\section*{Zpětná vazba (na cvičení)}
Tady a teď máš jedinečnou příležitost změnit předmět IB002 a~udělat jej pro
budoucí generace lepší. Využij této příležitosti a~zamysli se nad tím, co
se ti na cvičení líbilo, nelíbilo, co zachovat a~co změnit, a~napiš mi to.
Více návodných otázek najdeš na zadní straně, můžeš se jimi inspirovat.

\newpage

\section*{Zpětná vazba (na cvičení)}
Tady a teď máš jedinečnou příležitost změnit předmět IB002 a~udělat jej pro
budoucí generace lepší. Využij této příležitosti a~zamysli se nad tím, co
se ti na cvičení líbilo, nelíbilo, co zachovat a~co změnit, a~napiš mi to.
Více návodných otázek najdeš na zadní straně, můžeš se jimi inspirovat.

\newpage

\section*{Otázky}

Následující otázky můžeš chápat jako inspiraci pro psaní zpětné vazby na
druhou stranu. Rozhodně není třeba je jich striktně držet.

\begin{enumerate}
	\item Co se ti na Honzovi líbilo nejvíce – proč bys svého cvičícího
	doporučil/a ostatním studentům?
	\item Co by měl Honza naopak zlepšit/doučit se/natrénovat?
	\item Jak by vypadalo tvé ideální cvičení (např. míra opakování tématu,
		  samostatného procvičování, demonstračního řešení příkladů, ...)?
	\item Bavila tě práce ve skupinách? Vyhovovalo ti rozdělení učebny do
	      čtveřic?
	\item Na kolika domácích úlohách jsi aktivně pracoval?
	\item Kolik domácích úloh jsi úspěšně odevzdal?
	\item Jaký máš názor na témata, která se v~tomto předmětu probírala? Která
	      z~nich ti přišla zajímavá nebo užitečná? Která ti naopak přišla
	      nezajímavá nebo zbytečná? Přišla ti některá témata obtížná a~myslíš si,
	      že by bylo vhodné jim věnovat více času? \\
	      Pro připomenutí, probírali jsme: Korektnost, Složitost, Rekurze,
	      Řadicí algoritmy, Halda, Binární vyhledávací stromy, Červeno-černé
	      stromy, B-stromy, Hašovací tabulka, Průzkum grafů, Cesty v~grafech.
	\item Co je ta nejlepší věc, kterou sis ze cvičení odnesl?
\end{enumerate}

\newpage

\section*{Otázky}

Následující otázky můžeš chápat jako inspiraci pro psaní zpětné vazby na
druhou stranu. Rozhodně není třeba je jich striktně držet.

\begin{enumerate}
	\item Co se ti na Honzovi líbilo nejvíce – proč bys svého cvičícího
	doporučil/a ostatním studentům?
	\item Co by měl Honza naopak zlepšit/doučit se/natrénovat?
	\item Jak by vypadalo tvé ideální cvičení (např. míra opakování tématu,
		  samostatného procvičování, demonstračního řešení příkladů, ...)?
	\item Bavila tě práce ve skupinách? Vyhovovalo ti rozdělení učebny do
	      čtveřic?
	\item Na kolika domácích úlohách jsi aktivně pracoval?
	\item Kolik domácích úloh jsi úspěšně odevzdal?
	\item Jaký máš názor na témata, která se v~tomto předmětu probírala? Která
	      z~nich ti přišla zajímavá nebo užitečná? Která ti naopak přišla
	      nezajímavá nebo zbytečná? Přišla ti některá témata obtížná a~myslíš si,
	      že by bylo vhodné jim věnovat více času? \\
	      Pro připomenutí, probírali jsme: Korektnost, Složitost, Rekurze,
	      Řadicí algoritmy, Halda, Binární vyhledávací stromy, Červeno-černé
	      stromy, B-stromy, Hašovací tabulka, Průzkum grafů, Cesty v~grafech.
	\item Co je ta nejlepší věc, kterou sis ze cvičení odnesl?
\end{enumerate}

\end{document}
